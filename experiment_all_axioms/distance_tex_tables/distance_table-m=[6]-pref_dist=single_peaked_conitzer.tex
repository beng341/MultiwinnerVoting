
\begin{table*}
\centering
\begin{tabular}{lcccccccccccc}
\toprule
 & NN & Random & Borda & SNTV & Bloc & PAV & CC & lex-CC & seq-CC & Monroe & Greedy M. & MAV \\
\midrule
NN & -- & -- & -- & -- & -- & -- & -- & -- & -- & -- & -- & -- \\
Random & 0.389 & -- & -- & -- & -- & -- & -- & -- & -- & -- & -- & -- \\
Borda & 0.049 & 0.389 & -- & -- & -- & -- & -- & -- & -- & -- & -- & -- \\
SNTV & 0.351 & 0.389 & 0.353 & -- & -- & -- & -- & -- & -- & -- & -- & -- \\
Bloc & 0.133 & 0.389 & 0.169 & 0.303 & -- & -- & -- & -- & -- & -- & -- & -- \\
PAV & 0.165 & 0.389 & 0.168 & 0.320 & 0.045 & -- & -- & -- & -- & -- & -- & -- \\
CC & 0.383 & 0.389 & 0.393 & 0.354 & 0.264 & 0.240 & -- & -- & -- & -- & -- & -- \\
lex-CC & 0.178 & 0.389 & 0.175 & 0.325 & 0.058 & 0.014 & 0.237 & -- & -- & -- & -- & -- \\
seq-CC & 0.379 & 0.390 & 0.377 & 0.312 & 0.273 & 0.250 & 0.383 & 0.249 & -- & -- & -- & -- \\
Monroe & 0.358 & 0.389 & 0.368 & 0.342 & 0.238 & 0.215 & 0.050 & 0.214 & 0.354 & -- & -- & -- \\
Greedy M. & 0.219 & 0.389 & 0.209 & 0.313 & 0.112 & 0.079 & 0.276 & 0.071 & 0.209 & 0.254 & -- & -- \\
MAV & 0.341 & 0.388 & 0.346 & 0.409 & 0.339 & 0.339 & 0.198 & 0.339 & 0.489 & 0.243 & 0.369 & -- \\
\bottomrule
\end{tabular}

\caption{Distance between rules for 6 alternatives with $1 \leq k < m$ on on Single-peaked (Conitzer) preferences. Darker values correspond to larger distances. A distance of 0 between two rules indicates the rules always elect the same committee while a distance of 1 indicates that the rules' winning committees never have any overlap. Note that a distance of 1 is not possible when $k > \frac{m}{2}$ as committees must then overlap on some alternatives.}
\end{table*}
