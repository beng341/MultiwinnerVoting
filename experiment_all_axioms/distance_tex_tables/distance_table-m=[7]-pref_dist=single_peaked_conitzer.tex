
\begin{table*}
\centering
\begin{tabular}{lcccccccccccc}
\toprule
 & NN & Random & Borda & SNTV & Bloc & PAV & CC & lex-CC & seq-CC & Monroe & Greedy M. & MAV \\
\midrule
NN & -- & -- & -- & -- & -- & -- & -- & -- & -- & -- & -- & -- \\
Random & 0.381 & -- & -- & -- & -- & -- & -- & -- & -- & -- & -- & -- \\
Borda & 0.049 & 0.381 & -- & -- & -- & -- & -- & -- & -- & -- & -- & -- \\
SNTV & 0.348 & 0.381 & 0.349 & -- & -- & -- & -- & -- & -- & -- & -- & -- \\
Bloc & 0.133 & 0.381 & 0.160 & 0.306 & -- & -- & -- & -- & -- & -- & -- & -- \\
PAV & 0.166 & 0.381 & 0.165 & 0.319 & 0.046 & -- & -- & -- & -- & -- & -- & -- \\
CC & 0.370 & 0.381 & 0.375 & 0.358 & 0.281 & 0.266 & -- & -- & -- & -- & -- & -- \\
lex-CC & 0.182 & 0.381 & 0.173 & 0.324 & 0.062 & 0.016 & 0.266 & -- & -- & -- & -- & -- \\
seq-CC & 0.360 & 0.381 & 0.351 & 0.314 & 0.279 & 0.260 & 0.420 & 0.260 & -- & -- & -- & -- \\
Monroe & 0.344 & 0.381 & 0.349 & 0.348 & 0.263 & 0.249 & 0.057 & 0.251 & 0.399 & -- & -- & -- \\
Greedy M. & 0.216 & 0.381 & 0.206 & 0.314 & 0.112 & 0.081 & 0.302 & 0.075 & 0.223 & 0.286 & -- & -- \\
MAV & 0.331 & 0.381 & 0.335 & 0.401 & 0.331 & 0.332 & 0.171 & 0.331 & 0.490 & 0.207 & 0.362 & -- \\
\bottomrule
\end{tabular}

\caption{Distance between rules for 7 alternatives with $1 \leq k < m$ on on Single-peaked (Conitzer) preferences. Darker values correspond to larger distances. A distance of 0 between two rules indicates the rules always elect the same committee while a distance of 1 indicates that the rules' winning committees never have any overlap. Note that a distance of 1 is not possible when $k > \frac{m}{2}$ as committees must then overlap on some alternatives.}
\end{table*}
